\documentclass[10pt,twoside]{article}

\usepackage[margin=.8in]{geometry}
\usepackage{epsfig}
\usepackage{calc}
\usepackage{amssymb}
\usepackage{amstext}
\usepackage{amsmath}
\usepackage{amsthm}
\usepackage{multicol}
\usepackage{pslatex}
\usepackage{apalike}
\usepackage{graphicx}
\usepackage{caption}
\usepackage{subcaption}

\begin{document}

\title{6.867 Machine Learning  \subtitle{Homework 3} }

\maketitle

% **************************************************************************************************
 % Problem 1
% **************************************************************************************************

\section{\uppercase{Neural Networks}}

\noindent Neural networks are a highly successful approach to the classification problem in machine learning. In this section, we implement a feed-forward neural network from scratch and demonstrate its performance on a few simple datasets.

\subsection{Implementation}

\noindent 

\subsection{Section 2}


\subsubsection{sub sub sub section}


\subsection{Install Latex}

% **************************************************************************************************
 % Problem 2
% **************************************************************************************************

\section{\uppercase{Convolutional Neural Networks}}

\noindent The task of In this section, we experimented with different CNN architectures and 

\subsection{Design}

\noindent 

\subsection{Max Pooling}

\subsection{Regularization}

Some table. 
\begin{center}
 \begin{tabular}{||c c c c||} 
 \hline
  & Training Error & Validation Error & Test Error \\ [0.5ex] 
 \hline\hline
 Dataset 1 & 0 & 0 & 0 \\ 
 \hline
 Dataset 2 & 0.1775 & 0.18 & 0.195 \\
 \hline
 Dataset 3 & 0.02 & 0.03 & 0.045 \\
 \hline
 Dataset 4 & 0.3 & 0.305 & 0.3 \\
 \hline
\end{tabular}
\end{center}

\begin{figure}[h]
        \begin{subfigure}[b]{0.25\textwidth}
                \centering
                \includegraphics[width=\linewidth]{Figures/P2/svm_data1_test_C1.png}
                \caption{Dataset 1, C=1}
        \end{subfigure}%
        \begin{subfigure}[b]{0.25\textwidth}
                \centering
                \includegraphics[width=\linewidth]{Figures/P2/svm_data2_test_C1.png}
                \caption{Dataset 2, C=1}
        \end{subfigure}%
        \begin{subfigure}[b]{0.25\textwidth}
                \centering
                \includegraphics[width=\linewidth]{Figures/P2/svm_data3_test_C1.png}
                \caption{Dataset 3, C=1}
        \end{subfigure}%
        \begin{subfigure}[b]{0.25\textwidth}
                \centering
                \includegraphics[width=\linewidth]{Figures/P2/svm_data4_test_C1.png}
                \caption{Dataset 4, C=1}
        \end{subfigure}
        \caption{Linear SVM perfomance of different datasets. Datasets 1 and 3 are relatively linearly separable so this SVM performs well on them. Dataset 2 and 4 are definitely not linearly separable so this SVM is not the right model for the data.}\label{fig:animals}
\end{figure}

\subsection{Section}


\vfill
\end{document}

